%!TEX encoding =  IsoLatin
\input{../../common/exo_begin.tex}%

\firstpassagedo{
\newcommand{\sametextforthisinterro}[0]{ 
\huge ENSAE TD not�, mardi 3 novembre 2021

\normalsize
\textit{Le programme devra �tre envoy� par mail au charg� de TD et au professeur.}
\smallskip
}

\sametextforthisinterro
}


\exosubject{}
\begin{xexercice}\label{td_note_label1_2022}%\indexfrr{�nonc�}{pratique}

On s'int�resse � l'optimisation de la multiplication de plusieurs matrices.

\textbf{Pr�requis}

On a besoin d'une fonction qui mesure le temps d'ex�cution d'une fonction.

\begin{verbatimx}
import time

def mesure_temps_fonction(fct, N=100):
    begin = time.perf_counter()
    for i in range(N):
        fct()
    return (time.perf_counter() - begin) / N

mesure_temps_fonction(lambda: time.sleep(0.1), N=10)
\end{verbatimx}

\exequest Pourquoi \texttt{(m1 @ m2) @ m3} est-il plus lent que \texttt{m1 @ (m2 @ m3)} avec les matrices suivantes ? (2 points)

\begin{verbatimx}
import numpy

m1 = numpy.random.randn(997, 93)
m2 = numpy.random.randn(93, 1003)
m3 = numpy.random.randn(1003, 97)

mesure_temps_fonction(lambda: m1 @ m2 @ m3)
\end{verbatimx}

\exequest Ecrire une fonction qui calcule le nombre d'operations dans une multiplication de deux matrices (2 points)

\begin{verbatimx}
def n_ops(m1_shape, m2_shape):
    # ...
\end{verbatimx}

\exequest Ecrire une fonction qui retourne le meilleur co�t d'une multiplication de trois matrices et la meilleure option. (2 points)

\begin{verbatimx}
def n_ops_3(sh1, sh2, sh3):
    # ...
    return n_operation, (1 ou 2)
\end{verbatimx}

\exequest Ecrire une fonction qui effectue le produit de trois matrices le plus rapidement possible (2 points)

\begin{verbatimx}
def produit3(m1, m2, m3):
    #...
    return 
\end{verbatimx}

\exequest V�rifiez que vous retrouvez les m�mes r�sultats avec la fonction \texttt{mesure\_temps} (2 points)

\textbf{Ici, vous avez le choix entre faire les questions 6 � 9 ou les questions 9 et 10.}

\exequest Ecrire une fonction qui retourne le meilleur co�t d'une multiplication de 4 matrices et la meilleure option. La fonction doit appeler \texttt{n\_ops\_3}. (3 points)


\begin{verbatimx}
m4 = numpy.random.randn(97, 20)

def n_ops_4(sh1, sh2, sh3, sh4):
    # ...
    return n_operation, (1 ou 2 ou 3)
\end{verbatimx}

\exequest Ecrire une fonction qui effectue le produit de 4 matrices le plus rapidement possible. La fonction doit appeler \texttt{produit3}. (3 points)


\begin{verbatimx}
def produit4(m1, m2, m3, m4):
    # ...
		return 
\end{verbatimx}

\exequest V�rifiez que vous retrouvez les m�mes r�sultats avec la fonction \texttt{mesure\_temps} et la matrice m4. (2 points)

\exequest On se penche sur le cas � une multiplication de N matrices, combien y a-t-il de multiplications de 2 matrices ? (2 points)

\textbf{Ici s'arr�te l'�nonc� pour ceux qui ont choisit de r�pondre aux question 6 � 9. L'objectif de ces premi�res questions est de pr�parer le cas d'une multiplication de N matrices.}

\exequest R�soudre l'optimisation de multiplication de N matrices.

\end{xexercice}

\newpage






\input{../../common/exo_end.tex}%
