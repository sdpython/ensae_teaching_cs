%!TEX encoding =  IsoLatin
\input{../../common/exo_begin.tex}%

\firstpassagedo{
\huge ENSAE TD not�, mardi 27 novembre 2012

\normalsize
\textit{Le programme construit au fur et � mesure des questions devra �tre imprim� � la fin du TD et rendu au charg� de TD. \textbf{Il ne faut pas oublier de mentionner son nom ins�r� en commentaire au d�but du programme et de l'ajouter sur chaque page.} Les r�ponses autres que des parties de programme seront ins�r�es sous forme de commentaires. Les squelettes de fonctions propos�s ne sont que des suggestions. Les programmes fournis avec l'�nonc� ne devront pas �tre rendus. \textbf{Il faudra aussi indiquer le num�ro des exercices sous forme de commentaires.}} 
\smallskip
}



\exosubject{}
\begin{xexercice}\label{td_note_labelS_2013}%\indexfrr{�nonc�}{pratique}
L'objectif de cet exercice est de colorier l'espace situ� entre les deux spirales de la figure~\ref{double_spirale_2013}.


\begin{figure}[ht]
\begin{center}\begin{tabular}{|c|}\hline 
\includegraphics[width=8cm]{../python_examen/image/double_ellipse.png} \\ \hline
\end{tabular}
\end{center}
\caption{Double spirale}
\label{double_spirale_2013}
\end{figure}

Le code qui a servi � construire ces deux spirales vous est fourni en pi�ce jointe\footnote{lien \httpstyle{http://www.xavierdupre.fr/enseignement/complements\_site\_web/td\_note\_2013\_novembre\_2012\_exoM.py}}. Ce code vous est donn�, aucune question ne vous sera pos�e dessus. Il est pr�sent � la fin de l'�nonc� de l'exercice. \textbf{Afin d'�viter son inclusion (et son impression), votre programme devra imp�rativement commencer par~:}

\begin{verbatimx}
#coding:latin-1
import exoS
matrice = exoS.construit_matrice(100)
\end{verbatimx}

Pour visualiser la matrice et obtenir la figure~\ref{double_spirale_2013}, il faut �crire~:
\begin{verbatimx}
exoS.dessin_matrice(matrice)   # cette fonction place un point bleu pour une case contenant 1,
                               # rouge pour une case contenant 2,
                               # rien si elle contient 0
\end{verbatimx}

Au d�part la matrice retourn�e par la fonction \codes{construit\_matrice} contient soit~0 si la case est vide, soit~1 si cette case fait partie du trac� d'une spirale. L'objectif de cet exercice est de colorier une zone de la matrice.

\exequest Ecrire une fonction qui prend comme entr�e une matrice, deux entiers et qui retourne la liste des voisins parmi les quatre possibles pour lesquels la matrice contient une valeur nulle. On fera attention aux bords du quadrillage. (2~points)

\begin{verbatimx}
def voisins_a_valeur_nulle ( matrice, i, j ) :
    resultat = [ ]
    # ...
    return resultat
\end{verbatimx}


\exequest En utilisant la fonction pr�c�dente, �crire une fonction qui re�oit une liste de points de la matrice contenant une valeur nulle et qui retourne tous les voisins possibles contenant une valeur nulle pour tous les points de la liste. (2~points)

\begin{verbatimx}
def tous_voisins_a_valeur_nulle ( matrice, liste_points ) :
    resultat = [ ]
    # ...
    return resultat
\end{verbatimx}

\exequest On a tous les �l�ments pour �crire l'algorithme de coloriage~:
\begin{enumerate}
\item On part d'un point $(i0,j0)$ qui fait partie de la zone � colorier, qu'on ins�re dans une liste \codes{acolorier = [(i0,j0)]}.
\item Pour tous les points $(i,j)$ de la liste \codes{acolorier}, on change la valeur de la case $(i,j)$ de 0 � 2.
\item Pour tous les points $(i,j)$ de la liste \codes{acolorier}, on regarde les quatre voisins $(i,j+1), (i,j-1), (i+1,j), (i-1,j)$. Si la matrice contient 0 pour un de ces voisins, on l'ajoute � la liste et on retourne l'�tape pr�c�dente tant que cette liste n'est pas vide.
\end{enumerate}

En utilisant la fonction pr�c�dente, �crire une fonction qui colorie la matrice � partir d'un point $(i0,j0)$. (3~points)

\begin{verbatimx}
def fonction_coloriage( matrice, i0, j0) :
    # ...
\end{verbatimx}


\exequest Tester la fonction pr�c�dente avec le point de coordonn�es $(53,53)$ puis afficher le r�sultat avec la fonction \codes{dessin\_matrice}. (1~point)


\textbf{Il y avait deux questions diff�rentes selon les �nonces.}

\exequest Enonc�~1~: Ecrire une fonction qui retourne la surface colori�e. (2~points)
\begin{verbatimx}
def surface_coloriee ( matrice ) :
    # ...
    return surface
\end{verbatimx}

\exequest Enonc�~4~: Cr�er une autre fonction identique � celle de la question 3 � ceci pr�s qu'elle doit s'arr�ter apr�s que environ 1000 points distincts ont �t� colori�s. (2~points)
\begin{verbatimx}
def fonction_coloriage_environ_1000 ( matrice, i0, j0 ) :
    # ...
    return surface
\end{verbatimx}

\textbf{Programme g�n�rant la spirale}
\inputcodes{../python_examen/td_note_2013_novembre_2012_exoS.py}{g�n�ration des deux spirales}{, �nonc� 2013}
\end{xexercice}



\ifnum\correctionenonce = 1

\begin{xdemoexonot}{td_note_label_2013_S}

\inputcodes{../python_examen/td_note_2013.py}{exercice pour s'�valuer}{, correction 2013}



\end{xdemoexonot}
\fi




\exosubject{}
\begin{xexercice}\label{td_note_labelM_2013}%\indexfrr{�nonc�}{pratique}
\textbf{Certaines questions sugg�rent l'utilisation du module \codes{numpy}, ne pas le faire ne sera pas p�nalis� si les r�ponses propos�es produisent des r�sultats �quivalents. }

Cet exercice utilise des donn�es\footnote{lien \httpstyle{http://www.xavierdupre.fr/enseignement/complements\_site\_web/equipements\_sportif\_2011.zip}} fournies en pi�ce jointe. Elles comptabilisent le nombre d'�quipements sportifs par canton. Il faudra utiliser le programme �galement fourni en pi�ce jointe\footnote{lien \httpstyle{http://www.xavierdupre.fr/enseignement/complements\_site\_web/td\_note\_2013\_novembre\_2012\_exoM.py}} pour les r�cup�rer sous forme de tableau. Ce code ne sera pas modifi� durant l'examen except� le dernier param�tre transmis � la fonction \codes{construit\_matrice} qui permet de ne consid�rer qu'une partie des donn�es pour tester plus rapidement ses id�es sur les premi�res lignes. On souhaite mesurer si les Fran�ais ont acc�s aux m�mes �quipements sportifs sur tout le territoire fran�ais. \textbf{Afin d'�viter son inclusion (et son impression), votre programme devra imp�rativement commencer par~:}

\begin{verbatimx}
#coding:latin-1
import exoM
fichier_zip   = exoM.import_module_or_file_from_web_site ("equipements_sportif_2011.zip")
fichier_texte = exoM.unzip_fichier (fichier_zip)
# enlever le dernier param�tre 500 pour avoir le tableau complet
colonne, intitule, variables = exoM.construit_matrice (fichier_texte, 500)  
    # colonne   : contient le nom des colonnes
    # intitule  : contient les deux premi�res colonnes du fichier textes avec du texte
    # variables : contient les autres colonnes avec des valeurs num�riques 
\end{verbatimx}

Les lignes suivantes permettent de convertir les informations extraites en un tableau \codes{numpy}\footnote{Documentation~: \httpstyle{http://docs.scipy.org/doc/numpy/reference/}}.

\begin{verbatimx}
import numpy
intitule  = numpy.array(intitule)  # array et non matrix
variables = numpy.array(variables) # array et non matrix

# utilisation de numpy pour s�lectionner des lignes sp�cifiques
print intitule [ intitule[:,1] == "Chevroux", : ]  # affiche [['01102' 'Chevroux']]
print variables[ intitule[:,1] == "Chevroux", : ]  # affiche [[  82.    1.   12 ...
\end{verbatimx}

\exequest Le tableau \codes{intitule} a deux colonnes~: le code postal et la ville. On veut cr�er un tableau \codes{intitule3} qui contient trois colonnes~: celles de \codes{intitule} et le d�partement d�duit du code postal. Quelques fonctions utiles~: (2~points)
\begin{verbatimx}
print tab.shape                     # si tab est une matrice ou un tableau numpy � deux dimensions,
                                    # tab.shape est un couple (nb_lignes, nb_colonnes)
a = numpy.column_stack ( ( m, e ) ) # coller deux matrices, tableaux ayant le m�me nombre de lignes
\end{verbatimx}

Au final, la ligne \codes{['01008', 'Ambutrix']} deviendra \codes{['01008', 'Ambutrix', '01']}.



\exequest Construire la liste des d�partements, cette liste contient une unique instance du code du d�partement. Elle doit �tre tri�e par ordre croissant. Il n'est pas recommand� d'utiliser \codes{numpy} pour cette question. En anglais, tri se dit \textit{sort}. (2~points)

\exequest Construire un tableau $D$ de dimension $d \times v$ o� $d$~est le nombre de d�partements distincts, $v$~est le nombre de variables (normalement~105). Le coefficient $D_{ij}$ est la somme des valeurs pour la variable~$j$ et le d�partement~$i$. Si $A$ d�signe le tableau \codes{variables}, $B$ le tableau � trois colonnes de la question~1, $C$ la liste des d�partements distincts (question~2)~:
$$
D_{ij} = \sum_{k | B_{k,3} = C_i} A_{kj}
$$


L'objectif de cette question est d'agr�ger des donn�es par d�partements alors qu'elles sont disponibles par canton. (3~points)

\textbf{Remarque~:} l'instruction suivante pourrait �tre utile.

\begin{verbatimx}
# cr�e une matrice de dimension nb_lignes x nb_colonnes initialis�s � z�ro
mvide = numpy.zeros ( ( nb_lignes, nb_colonnes) )
\end{verbatimx}



\exequest La colonne~5 du tableau~$D$ (la premi�re colonne a l'indice~0) contient la population. Cr�er un autre tableau~$E$ qui v�rifie~: $E_{ij} = D_{ij} / D_{i5}$. (1~point)

\exequest Le programme fourni en pi�ce jointe contient une fonction \codes{coefficient\_gini} qui calcule le coefficient de Gini\footnote{\httpstyle{http://fr.wikipedia.org/wiki/Coefficient\_de\_Gini}}. On l'utilise pour comparer le nombre d'�quipements par habitants. Il vaut~0 si ce ratio est constant quelque soit le d�partement, il vaut~1 si un seul d�partement propose un certain type d'�quipement. Entre~0 et~1, il indique l'in�galit� de la distribution. Quel est l'�quipement sportif le plus in�galitairement r�parti sur tout le territoire~? (2~points)

\textbf{Remarque~:} les lignes suivantes pourront vous aider.

\begin{verbatimx}
li = list ( mat [:,i] )           # convertit une colonne d'un tableau numpy en une liste
print colonne[0][i+2]             # affiche le label de la colonne i
gini = exoM.coefficient_gini (li) # retourne le coefficient de Gini
                                  # pour la liste li
\end{verbatimx}




\medskip

\textbf{Programme pr�parant les donn�es~:}
\inputcodes{../python_examen/td_note_2013_novembre_2012_exoM.py}{import des donn�es}{, �nonc� 2013}
\end{xexercice}





\ifnum\correctionenonce = 1

\begin{xdemoexonot}{td_note_label_2013_M}

\inputcodes{../python_examen/td_note_2013_M.py}{exercice pour s'�valuer}{, correction 2013}



\end{xdemoexonot}
\fi


\input{../../common/exo_end.tex}%
