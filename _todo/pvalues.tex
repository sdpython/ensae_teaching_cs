\input{../../common/paper_begin.tex}%

\firstpassagedo{
\title{Relationship between p-values and confidence intervals}
\author{Xavier Dupr� \\ \httpstyle{http://www.xavierdupre.fr/}}
\maketitle
\begin{abstract}
\noindent This document explains the relationship between p-value and confidence intervals. It goes on with the specific case of a binamial law. Assuming we want to determine whether or not two binomial laws are significantly different, how many observations we need to get the p-value under a given threshold. 
\end{abstract}
\setcounter{tocdepth}{2}
\tableofcontents
\hypersetup{
    pdftitle={Relationship between p-values and confidence intervals}
    pdfauthor={Xavier Dupr�}
    pdfsubject={Relationship between p-values and confidence intervals}
    pdfkeywords={p-value confidence interval}
}
\renewcommand{\thexctheorem}    {\arabic{xctheorem}}
%\renewcommand{\thesection}{\arabic{section}}
\renewcommand{\thesubsection}{\arabic{subsection}}

}

\begin{xexempleprog}{Confidence Interval and p-Value}{exemple_pvalues_cor}\label{exemple_pvalues}
\indexfr{p-value}
\indexfr{intervalle de confiance}

\subsubsectionx{Confidence interval}


The term \textit{p-value} is very popular in the world of search engines. I usually prefer confidence interval 95\%, I think it is easier to understand. Plus, because p-Value are real values, we could be tempted to compare them and it is usually wrong. On the other hand, it is more difficult to compare confidence intervals, especially if they are related to complete different variables. Their nature prevents you from doing that. Howerver p-Values and confidence interval are similar: they tell you whether or not a metric difference is significant.

Usually, it starts from a set of identically distributed random variables $(X_i)_{1 \infegal i \infegal N}$. We then estimate the average $\widehat{\theta}_N = \frac{1}{N} \sum_{i=1}^{N} X_i$ and we ask the question: is $\widehat{\theta}_N$ null? In others terms, we want to know if the average is significantly different from zero. If the random variable $X$ follows a random law which has a standard deviation\footnote{Not all of them have a standard deviation. For example, if $X$ follows a Cauchy law, $\esp{X^2} \sim \int \frac{x^2}{1+x^2}dx$ which does not exist. This remark also concerns every distribution known as heavy tail distribution.}, we can use the central limit theorem\footnote{See \httpstyle{http://en.wikipedia.org/wiki/Central\_limit\_theorem}.} which tells us:

\begin{eqnarray}
\sqrt{N} \widehat{\theta}_N \underset{N \rightarrow \infty}{\longrightarrow}   \loinormale{0}{\sigma}
\end{eqnarray}

If $Y \sim \loinormale{0}{\sigma}$, then we have $\pr{\abs{Y} \infegal 1.96} = 0.95$. That is why we can say:

\begin{eqnarray}
\widehat{\theta}_N \text{ is not null with 95\% confidence if } \sqrt{N} \frac{|\widehat{\theta}_N|}{\sigma} > 1.96 
\end{eqnarray}

And the confidence intervalle at 95\% would be: 

\begin{eqnarray}
\cro{ %\widehat{\theta}_N 
			- \frac{1.96 \sigma}{\sqrt{N}}, 
			%\widehat{\theta}_N + 
			\frac{1.96 \sigma}{\sqrt{N}}}
\end{eqnarray}

When $\esp{ \widehat{\theta}_N } = \theta_0 \neq 0$, it becomes:

\begin{eqnarray}
\sqrt{N} \cro{ \widehat{\theta}_N - \theta_0} \underset{N \rightarrow \infty}{\longrightarrow}   \loinormale{0}{\sigma}
\end{eqnarray}

We usually want to check if the mean is equal to a specific value using a statistical test: 
\begin{eqnarray*}
H0: && \widehat{\theta}_N = \theta_0 \\
H1: && \widehat{\theta}_N \neq  \theta_0 
\end{eqnarray*}

We validate $H0$ if:

\begin{eqnarray}
\widehat{\theta}_N \in \cro{ \theta_0 - \frac{1.96 \sigma}{\sqrt{N}}, \theta_0 + \frac{1.96 \sigma}{\sqrt{N}}}
\end{eqnarray}



\subsubsectionx{p-value}

With confidence intervals, you first choose a confidence level and then you get an interval. You then check if your value is inside or outside your interval. Inside, the gain is not significant, outside, it is. 

With a p-value, we consider the problem the other way: given $\widehat{\theta}_N$, what is the probability that the difference $\abs{\widehat{\theta}_N - \theta_0}$ is significant? Let's consider $Y$ following a normal law $\loinormale{0}{1}$. We are looking for:

\begin{eqnarray}
\pr{ \abs{Y} >  \sqrt{N} \frac{|\widehat{\theta}_N|}{\sigma} }  = \alpha
\end{eqnarray}

$\alpha$ is the p-value.

\begin{eqnarray}
\alpha &=& 1-\int_{-\beta_N}^{\beta_N} \frac{1}{\sqrt{2\pi}} e^{\frac{-x^2}{2}} dx = 
							2 \int_{\beta_N}^{\infty} \frac{1}{\sqrt{2\pi}} e^{\frac{-x^2}{2}} dx \label{p_value_expression}\\
\text{where } \beta_N &=& \sqrt{N} \frac{|\widehat{\theta}_N|}{\sigma} \nonumber
\end{eqnarray}

At this point, we should not forget that we use a theorem which tells us that $\sqrt{N} \frac{|\widehat{\theta}_N|}{\sigma} \sim \loinormale{0}{1}$ when $N \rightarrow \infty$, which means everything we said is true when $N$ is great enough.





\subsubsectionx{Significant difference between samples mean}

Usually, we do not want to test if an average is null but if the difference between two averages is null. We consider two random samples having the same size, each of them described by $(X_i)$ and $(Y_i)$. All variables are independant. $(X_i)$ are distributed according the same law, we assume the same for $(Y_i)$. We expect the following difference to be null.

\begin{eqnarray}
\widehat{\eta}_N &=& \frac{1}{N} \sum_{i=1}^{N} X_i - \frac{1}{N} \sum_{i=1}^{N} Y_i 
							 = \frac{1}{N} \cro{ \sum_{i=1}^{N} X_i -  Y_i } \label{pvalues_exp2}
\end{eqnarray}

Considering expression~(\ref{pvalues_exp2}), we can applying the central limit theorem on variable $Z=X-Y$, we get ($eta_0=0$):

\begin{eqnarray}
\sqrt{N} \widehat{\eta}_N \underset{N \rightarrow \infty}{\longrightarrow}   \loinormale{\eta_0}{\sqrt{ \frac{\var{Z} }{N  } }}
\end{eqnarray}

If both samples do not have the same number of observations, this expression becomes:


\begin{eqnarray}
\sqrt{N} \widehat{\eta}_N \underset{ \begin{subarray}{c} N_1 \rightarrow \infty \\
																											 N_2 \rightarrow \infty \\
																											 \frac{N_1}{N_2} \rightarrow x \end{subarray}
 }{\longrightarrow}   \loinormale{\eta_0}{\sqrt{ \frac{\var{X}}{N_1}+\frac{\var{Y}}{N_2} }}
\end{eqnarray}


\subsubsectionx{Application on binomial variables}\label{section_pvalues_table}

A binomial variable $X \sim \loibinomialea{p}$ is defined by:

\begin{eqnarray*}
\pr{X=0} &=& 1-p \\
\pr{X=1} &=& p
\end{eqnarray*}

Let's consider two series of observations $(X_i) \sim \loibinomialea{p}$ and $(Y_i) \sim \loibinomialea{q}$. We assume $p \neq q$ and we want to determine how many observations we need to get a p-value below 5\%. We know that $\variance{X_i}=p(1-p)$ and $\variance{Y_i}=q(1-q)$. Table ~\ref{figure_section_pvalues_table} shows the values. First column contains values for $p$, first row contains values for $q-p$. We also assume we have the same number $N$ of random observations for each variable. The statistical test cen be defined like following:


\begin{eqnarray*}
H0: && p = q = p_0 \\
H1: && p \neq q 
\end{eqnarray*}

If H0 is true, then:

\begin{eqnarray}
\sqrt{N} \widehat{\theta}_N \underset{N \rightarrow \infty}{\longrightarrow}   \loinormale{0}{\sqrt{p_0(1-p_0)} 
																																													\sqrt{ \frac{1}{N_1}+\frac{1}{N_2} }}
\end{eqnarray}




\begin{table}[ht]
\begin{center}\begin{tiny}
\begin{tabular}{r|r|r|r|r|r|r|r|r|r|r|r|r}
$p/d$&\textbf{-0.200}&\textbf{-0.100}&\textbf{-0.020}&\textbf{-0.010}&\textbf{-0.002}&\textbf{-0.001}&\textbf{0.001}&\textbf{0.002}&\textbf{0.010}&\textbf{0.020}&\textbf{0.100}&\textbf{0.200}\\\hline
\textbf{0.05}&&&913&3650&91235&364939&364939&91235&3650&913&37&10\\\hline
\textbf{0.10}&&70&1729&6915&172866&691463&691463&172866&6915&1729&70&18\\\hline
\textbf{0.15}&&98&2449&9796&244893&979572&979572&244893&9796&2449&98&25\\\hline
\textbf{0.20}&31&123&3074&12293&307317&1229267&1229267&307317&12293&3074&123&31\\\hline
\textbf{0.25}&37&145&3602&14406&360137&1440548&1440548&360137&14406&3602&145&37\\\hline
\textbf{0.30}&41&162&4034&16135&403354&1613413&1613413&403354&16135&4034&162&41\\\hline
\textbf{0.35}&44&175&4370&17479&436966&1747864&1747864&436966&17479&4370&175&44\\\hline
\textbf{0.40}&47&185&4610&18440&460976&1843901&1843901&460976&18440&4610&185&47\\\hline
\textbf{0.45}&48&191&4754&19016&475381&1901523&1901523&475381&19016&4754&191&48\\\hline
\textbf{0.50}&49&193&4802&19208&480183&1920730&1920730&480183&19208&4802&193&49\\\hline
\textbf{0.55}&48&191&4754&19016&475381&1901523&1901523&475381&19016&4754&191&48\\\hline
\textbf{0.60}&47&185&4610&18440&460976&1843901&1843901&460976&18440&4610&185&47\\\hline
\textbf{0.65}&44&175&4370&17479&436966&1747864&1747864&436966&17479&4370&175&44\\\hline
\textbf{0.70}&41&162&4034&16135&403354&1613413&1613413&403354&16135&4034&162&41\\\hline
\textbf{0.75}&37&145&3602&14406&360137&1440548&1440548&360137&14406&3602&145&37\\\hline
\textbf{0.80}&31&123&3074&12293&307317&1229267&1229267&307317&12293&3074&123&31\\\hline
\textbf{0.85}&25&98&2449&9796&244893&979572&979572&244893&9796&2449&98&\\\hline
\textbf{0.90}&18&70&1729&6915&172866&691463&691463&172866&6915&1729&70&\\\hline
\textbf{0.95}&10&37&913&3650&91235&364939&364939&91235&3650&913&&\\\hline
\end{tabular}
\end{tiny}
\end{center}
\caption{Given a binomial law with parameter $p$ and a difference $d$, this table gives the number of 
observations needed on both sides to get a significant difference assuming $p$ is the expected pourcentage
(see program section~\ref{program_section_pvalues_table}, page~\pageref{program_section_pvalues_table}).}
\label{figure_section_pvalues_table}
\end{table}


\subsubsectionx{Estimate a p-value by using the distribution function}%\label{}

Expresion~(\ref{p_value_expression}) gives a way to estimate the p-value. Computing the integral is not always possible but there is a way to do it using Monte Carlo method. Let's assume $X \sim \loinormale{0}{1}$. We denote $f_X$ as the density function of $X$. We also consider an interval $I=\cro{-a,a}$. Then we have $f(a)=f(-a)$ and:

\begin{eqnarray}
\pr{X \in I} = \pr{ \abs{X} \infegal a } = \pr{ f(X) \supegal f(a)}
\end{eqnarray}

This is true because $f$ is decreasing for $x>0$. The p-value $\alpha$ for a estimator $\beta$ using Monte Carlo method is:

\begin{eqnarray}
\frac{1}{N}\sum_{i=1}^{N} \indicatrice{ f(X_i) < f(\beta)} \longrightarrow \alpha 
\end{eqnarray}

Assuming every $(X_i)_i$ follows a normal law $\loinormale{0}{1}$.





\subsubsectionx{Correlated variables}%\label{}

Let's assume we now have a vector a correlated variables~$X=(X_1,...X_d)$ drawn following a law $\loinormale{\theta_0}{\Sigma}$. 


The central limit theorem is still valid:

\begin{eqnarray}
\sqrt{N} \widehat{\theta}_N \underset{N \rightarrow \infty}{\longrightarrow}   \loinormale{\theta_0}{\Sigma}
\end{eqnarray}

We know estimators for the average and the covariance matrix defined as follows:

\begin{eqnarray}
\widehat{\theta_N} &=& \frac{1}{n} \sum_{i=1}^{N} X_i \\
\widehat{\Sigma_N} &=& \frac{1}{n} \sum_{i=1}^{N} (X_i - \widehat{\theta_N})(X_i - \widehat{\theta_N})'
\end{eqnarray}


We usually want to check if: 
\begin{eqnarray*}
H0: && \widehat{\theta}_N = \theta_0 \\
H1: && \widehat{\theta}_N \neq  \theta_0 
\end{eqnarray*}

If $\Lambda$ is diagonal matrix of $\Sigma$ (diagonal matrix with eigen values). All eigne values are real and positive, we then define:

\begin{eqnarray}
\Sigma &=& P \Lambda P' \text { and } \Sigma^{\frac{1}{2}} =  P \Lambda^{\frac{1}{2}} P' 
\end{eqnarray}


We consider $Z_i = (X_i - \widehat{\theta_N}) \Sigma^{-\frac{1}{2}}$. We then have: $\esp{Z_i} = 0$ and $\var{Z_i} = I_2$ where $I_2$ is the identity matrix. We could now consider each dimension of $Z_i$ independently as illustrated in Figure~\ref{pvalues_cor_area}: it shows the difference on an example if we consider the correlation of two variables correlated such as $\Sigma=\pa{\begin{array}{cc} 0.1 & 0.05 \\ 0.05 &  0.2 \end{array}}$. 

\begin{figure}[ht]
\figureoneimage{ \caption{	We assume we observe two Bernouilli variables correlated. Red points represents the area 
												for which we would accept hypothesis H0 in case both variables are independant.
												Blue area represents the same but with the correlation. See Section~\ref{pvalues_cor_area_prog}.} }
{ \includegraphics[width=8cm]{../python_divers/image/pvaluescor.png} }
{\label{pvalues_cor_area}}
\end{figure}
    		    
But that would not be the best way to do it. The confidence interval for a couple of indenpendant gaussian $(N_1,N_2)$ variables is an ellipse. Two independent normal $N_1^2+N_2^2$ with a null mean and standard deviation equal to one follows a $\chi_2$ law. Based on that, we can deduce a boundary for the confidence zone at 95\%. Figure~\ref{pvalues_cor_area2} shows this zone for a non-correlated couple and a correlated couple ($\Sigma=\pa{\begin{array}{cc} 0.1 & 0.05 \\ 0.05 &  0.2 \end{array}}$).

\begin{figure}[ht]
\figureoneimage{ \caption{	We assume we observe two Bernouilli variables correlated. Red points represents the area 
												for which we would accept hypothesis H0 in case both variables are independant.
												Blue area represents the same but with the correlation. See Section~\ref{pvalues_cor_area_prog}.} }
{ \includegraphics[width=8cm]{../python_divers/image/pvaluescor2.png} }
{\label{pvalues_cor_area2}}
\end{figure}
    		    














\if 0

\subsubsectionx{Algorithm Expectation-Maximization}\label{section_pvalues_table_em}

We here assume there are two populations mixed defined by random variable~$C$. Let's $X$ be a mixture of two binomial laws of parameters $p$ and $q$. It is for example the case for a series draws coming from two different coins.

\begin{eqnarray}
\pr{X} &=&  \pr{ X | C = a} \pr{C=a} + \pr{X | X =b} \pr{C = b}
\end{eqnarray}

The likelihood of a random sample $(X_1,...X_n)$, the class we do not observe are $(C_1,...C_n)$:

\begin{eqnarray}
L(\theta) = \prod_i \cro{ p^{X_i}(1-p)^{(1-X_i)} \pi }^{1-C_i} \cro{q^{X_i}(1-q)^{(1-X_i)}  (1-\pi)  }^{C_i}
\end{eqnarray}

The parameters are $\theta=(\pi,p,q)$. We use an algorithm Expectation-Maximization (EM) to determine the parameters. We define at iteration~$t$:

\begin{eqnarray}
w_i &=& \espf{C_i | X_i, \theta_t }{X_i}  \\
    &=& \frac{ p_t^{X_i} (1-p_t)^{1-X_i} \pi_t }
               { p_t^{X_i}(1-p_t)^{1-X_i} \pi_t + q_t^{X_i} (1-q_t)^{1-X_i} (1-\pi_t) }
\end{eqnarray}

We then update the parameters:

\begin{eqnarray} 
\widehat{\pi} &=&  \frac{1}{n} \sum_{i = 1}^n w_i \\ 
\widehat{p} 	&=&  \frac{  \sum_{i = 1}^n w_i X_i }{  \sum_{i = 1}^n w_i} \\ 
\widehat{q} 	&=&  \frac{  \sum_{i = 1}^n (1-w_i) X_i }{  \sum_{i = 1}^n (1-w_i)} 
\end{eqnarray}




%http://statisticalrecipes.blogspot.fr/2012/04/applying-em-algorithm-binomial-mixtures.html

\fi




\end{xexempleprog}

%------------------------------------------------------------------------------------------------------


\begin{xexempleprogcor}{exemple_pvalues}\label{exemple_pvalues_cor}


\subsubsectionx{Program which produces the table in Section~\ref{section_pvalues_table}}\label{program_section_pvalues_table}
\inputcode{../python_divers/pvalues/pvalues.py}{to be filled later}

\subsubsectionx{Program which generates Figure~\ref{pvalues_cor_area}}\label{pvalues_cor_area_prog}
\inputcode{../python_divers/pvalues/pvalues_sigma.py}{to be filled later}

\subsubsectionx{Program which estimates the p-value}\label{pvalues_estim_cor}
\inputcode{../python_divers/pvalues/pvalues_estimrnd.py}{to be filled later}


\end{xexempleprogcor}



%-----------------------------------------------------------------------------------------------------
% afin d'�viter d'inclure plusieurs ce fichier
%-----------------------------------------------------------------------------------------------------

\ifnum\nbpassages=1


%-----------------------------------------------------------------------------------------------------
% �crit la table des mati�res de fa�on d�taill�e
%-----------------------------------------------------------------------------------------------------
%\tableofcontents%
%\shorttableofcontents{Table des mati�res d�taill�e}{3}

%-----------------------------------------------------------------------------------------------------
% �crit l'index
%-----------------------------------------------------------------------------------------------------
%\printindex

%-----------------------------------------------------------------------------------------------------
\end{document}
%-----------------------------------------------------------------------------------------------------


\else


\ifnum\nbpassages=2
\nbpassages=1
\fi 

\ifnum\nbpassages=3
\nbpassages=2
\fi 

\ifnum\nbpassages=4
\nbpassages=3
\fi 

\ifnum\nbpassages=5
\nbpassages=4
\fi 


\fi
%
